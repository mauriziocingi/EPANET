\documentclass[a4paper,11pt]{Report}
\usepackage{latexsym}
\usepackage[english]{babel}
%\usepackage[utf8]{inputenc}
\usepackage[]{graphicx}
\usepackage{textcomp}
\usepackage{multirow}
\usepackage{longtable}
\usepackage{appendix}
\usepackage{listings}


\author{M.~Cingi}
\title{Epanet2 user's manual}


\begin{document}
\maketitle

\tableofcontents

\listoftables


\chapter{General Instructions}

EPANET can also be run as a console application from the command line within a
DOS window. In this case network input data are placed into a text file and results
are written to a text file. The command line for running EPANET in this fashion is:
epanet2d inpfile rptfile outfile
Here inpfile is the name of the input file, rptfile is the name of the output
report file, and outfile is the name of an optional binary output file that stores
results in a special binary format. If the latter file is not needed then just the input and
report file names should be supplied. As written, the above command assumes that
you are working in the directory in which EPANET was installed or that this
directory has been added to the PATH statement in your AUTOEXEC.BAT file.
Otherwise full pathnames for the executable epanet2d.exe and the files on the
command line must be used. The error messages for command line EPANET  are listed in Appendix B.


\chapter{Epanet toolkit}

EPANET is a program that analyzes the hydraulic and water quality behavior of
water distribution systems. The EPANET Programmer's Toolkit is a dynamic link 
library (DLL) of functions that allows developers to customize EPANET's computational 
engine for their own specific needs. The functions can be incorporated into 32-bit Windows 
applications written in C/C++, Delphi Pascal, Visual Basic, or any other language that can 
call functions within a Windows DLL. The Toolkit DLL file is named EPANET2.DLL and is 
distributed with EPANET. The Toolkit comes with several different header files, 
function definition files, and .lib files that simplify the task of interfacing 
it with C/C++, Delphi, and Visual Basic code.

function added in wateranalytics:
ENaddcurve, ENgetaveragepatternvalue, ENgetbasedemand, ENgetcoord, ENgetcurveid, ENgetcurveindex, 
ENgetcurvelen, ENgetcurve, ENgetcurvevalue, ENgetdemandpattern, ENgetflowunits, ENgetheadcurve, 
ENgetnumdemands, ENgetpumptype, ENgetqualinfo, ENgetstatistic, ENgettimeparam, ENsetbasedemand, 
ENsetcoord, ENsetcurve, ENsetcurvevalue





\begin{longtable}{|c| l|}
\hline
TASK&FUNCTION \\
\hline
\hline
\multirow{1}{*}{Running a complete simulation}
&ENepanet\\
\hline
\multirow{2}{*}{Opening and closing the EPANET Toolkit system}
&ENopen \\
&ENclose \\
\hline
\multirow{4}{*}{Retrieving information about network nodes}
&ENgetnodeindex\\
&ENgetnodeid\\
&ENgetnodetype\\
&ENgetnodevalue\\
\hline
\multirow{5}{*}{Retrieving information about network links}
&ENgetlinkindex\\
&ENgetlinkid\\
&ENgetlinktype\\
&ENgetlinknodes\\
&ENgetlinkvalue\\
\hline
\multirow{4}{*}{Retrieving information about time patterns}
&ENgetpatternid\\
&ENgetpatternindex\\
&ENgetpatternlen\\
&ENgetpatternvalue\\
\hline
\multirow{3}{*}{Retrieving other network information}
&ENgetcount\\
&ENgetcontrol\\
&ENgetqualtype\\
&ENgetoption\\
\hline
\multirow{9}{*}{Setting new values for network parameters}
&ENsetcontrol\\
&ENsetnodevalue\\
&ENsetlinkvalue\\
&ENaddpattern\\
&ENsetpattern\\
&ENsetpatternvalue\\
&ENsetqualtype\\
&ENsettimeparam\\
&ENsetoption\\
\hline
\multirow{2}{*}{Saving and using hydraulic analysis results files}
&ENsavehydfile\\
&ENusehydfile\\
\hline
\multirow{6}{*}{Running a hydraulic analysis}
&ENsolveH\\
&ENopenH\\
&ENinitH\\
&ENrunH\\
&ENnextH\\
&ENcloseH\\
\hline
\multirow{7}{*}{Running a water quality analysis}
&ENsolveQ\\
&ENopenQ\\
&ENinitQ\\
&ENrunQ\\
&ENnextQ\\
&ENstepQ\\
&ENcloseQ\\
\hline
\multirow{8}{*}{Generating an output report}
&ENsaveH\\
&ENsaveinpfile\\
&ENreport\\
&ENresetreport\\
&ENsetreport\\
&ENsetstatusreport\\
&ENgeterror\\
&ENwriteline\\
\hline
\end{longtable}


\section{toolkit code information}
 \subsection{ENgetversion}
\subsubsection{Declaration}
\begin{lstlisting}
int ENgetversion(int *version);
\end{lstlisting}
\subsubsection{Description}
retrieves a number assigned to the most recent
update of the source code. This number, set by the
constant CODEVERSION found in TYPES.H, is to be
interpreted with implied decimals, i.e., "20100" == "2(.)01(.)00"
\subsubsection{Arguments}
\begin{tabular}{l r p{11cm} }
int pointer&version&version number of the source code\\[6pt]
\end{tabular}
\subsubsection{Returns}
Returns an error code.
\subsubsection{Notes}
Error code should always be 0

\section{Running a complete simulation}
 \subsection{ENepanet}
\subsubsection{Declaration}
\begin{lstlisting}
int ENepanet(char *inpFile, char *rptFile, char *binOutFile, void (*callback) (char *));
\end{lstlisting}
\subsubsection{Description}
Runs a complete EPANET simulation.
\subsubsection{Arguments}
\begin{tabular}{l r p{11cm} }
char pointer&inpFile&name of the input file \\[6pt]
char pointer&rptFile&name of an output report file  \\[6pt]
char pointer&binOutFile&name of an optional binary output file \\[6pt]
function pointer& callback & user-supplied function which accepts a character string as its argument. \\[6pt]
\end{tabular}
\subsubsection{Returns}
Returns an error code.
\subsubsection{Notes}
ENepanet is a stand-alone function and does not interact with any of the other functions in the toolkit.
If there is no need to save EPANET's binary output file then binOutFile can be an empty string ("").
 The callback function pointer allows the calling program to display a progress message generated by EPANET 
 during its computations. A typical function for a console application might look as follows:
\begin{lstlisting}
void  writecon(char *s)
 {
    puts(s);
 }
\end{lstlisting}
  
and somewhere in the calling program the following declarations would appear:   
\begin{lstlisting}
 void (* vfunc) (char *);
 vfunc = writecon;
 ENepanet(f1,f2,f3,vfunc);
 \end{lstlisting}

If such a function is not desired then this argument should be NULL (NIL for Delphi/Pascal, 
 VBNULLSTRING for Visual Basic, None for Python).

 ENepanet is used mainly to link the EPANET engine to third-party user interfaces that build network 
 input files, read binary output file and display the results of a network analysis.

\section{Opening and closing the EPANET Toolkit system}
 \subsection{ENopen}
\subsubsection{Declaration}
\begin{lstlisting}
int ENopen( char* f1, char* f2, char* f3)
\end{lstlisting}

\subsubsection{Description}
Opens the Toolkit to analyze a particular distribution system.
\subsubsection{Arguments}
\begin{tabular}{ r p{11cm} }
f1&name of the input file \\[6pt]
f2& name of an output report file \\[6pt]
f3& name of an optional binary output file\\[6pt]
\end{tabular}

 
 
  
\subsubsection{Returns}
Returns an error code.
 


\subsubsection{Notes}
If there is no need to save EPANET's binary Output file then f3 can be an empty string ("").

ENopen must be called before any of the other toolkit functions (except ENepanet) are used.
 
\subsubsection{See also}
ENclose 

 \subsection{ENclose}
\subsubsection{Declaration}
\begin{lstlisting}
int ENclose( void )
\end{lstlisting}

\subsubsection{Description}
Closes down the Toolkit system (including all files being processed).
  
\subsubsection{Returns}
Returns an error code.
 


\subsubsection{Notes}
ENclose must be called when all processing has been completed, even if an error condition was encountered.  

\subsubsection{See also}
ENopen


\section{Retrieving information about network nodes}
 \subsection{ENgetnodeindex}

\subsubsection{Declaration}
\begin{lstlisting}
int ENgetnodeindex( char* id, int* index )
\end{lstlisting}

\subsubsection{Description}
Retrieves the index of a node with a specified ID.
\subsubsection{Arguments}
\begin{tabular}{ r p{11cm} }
id& node ID label   \\[6pt]
index&node index   \\[6pt]
\end{tabular}

  
\subsubsection{Returns}
Returns an error code.



\subsubsection{Notes}
Node indexes are consecutive integers starting from 1.
 
\subsubsection{See also}
ENgetnodeid

 \subsection{ENgetnodeid}
\subsubsection{Declaration}
\begin{lstlisting}
int ENgetnodeid(int index, char *id);
\end{lstlisting}
\subsubsection{Description}
** insert description here **
\subsubsection{Arguments}
\begin{tabular}{l r p{11cm} }
int&index&*insert arg description here* \\[6pt]
char pointer&id&*insert arg description here* \\[6pt]
\end{tabular}
\subsubsection{Returns}
Returns an error code.
\subsubsection{Notes}
* insert notes here *
 \subsection{ENgetnodetype}
\subsubsection{Declaration}
\begin{lstlisting}
int ENgetnodetype(int index, int *code);
\end{lstlisting}
\subsubsection{Description}
** insert description here **
\subsubsection{Arguments}
\begin{tabular}{l r p{11cm} }
int&index&*insert arg description here* \\[6pt]
int pointer&code&*insert arg description here* \\[6pt]
\end{tabular}
\subsubsection{Returns}
Returns an error code.
\subsubsection{Notes}
* insert notes here *
 \subsection{ENgetnodevalue}
\subsubsection{Declaration}
\begin{lstlisting}
int ENgetnodevalue(int index, int code, float *value);
\end{lstlisting}
\subsubsection{Description}
** insert description here **
\subsubsection{Arguments}
\begin{tabular}{l r p{11cm} }
int&index&*insert arg description here* \\[6pt]
int&code&*insert arg description here* \\[6pt]
float pointer&value&*insert arg description here* \\[6pt]
\end{tabular}
\subsubsection{Returns}
Returns an error code.
\subsubsection{Notes}
* insert notes here *

\section{Retrieving information about network links}
 \subsection{ENgetlinkindex}
\subsubsection{Declaration}
\begin{lstlisting}
int ENgetlinkindex(char *id, int *index);
\end{lstlisting}
\subsubsection{Description}
** insert description here **
\subsubsection{Arguments}
\begin{tabular}{l r p{11cm} }
char pointer&id&*insert arg description here* \\[6pt]
int pointer&index&*insert arg description here* \\[6pt]
\end{tabular}
\subsubsection{Returns}
Returns an error code.
\subsubsection{Notes}
* insert notes here *
 \subsection{ENgetlinkid}
\subsubsection{Declaration}
\begin{lstlisting}
int ENgetlinkid(int index, char *id);
\end{lstlisting}
\subsubsection{Description}
** insert description here **
\subsubsection{Arguments}
\begin{tabular}{l r p{11cm} }
int&index&*insert arg description here* \\[6pt]
char pointer&id&*insert arg description here* \\[6pt]
\end{tabular}
\subsubsection{Returns}
Returns an error code.
\subsubsection{Notes}
* insert notes here *
 \subsection{ENgetlinktype}
\subsubsection{Declaration}
\begin{lstlisting}
int ENgetlinktype(int index, int *code);
\end{lstlisting}
\subsubsection{Description}
** insert description here **
\subsubsection{Arguments}
\begin{tabular}{l r p{11cm} }
int&index&*insert arg description here* \\[6pt]
int pointer&code&*insert arg description here* \\[6pt]
\end{tabular}
\subsubsection{Returns}
Returns an error code.
\subsubsection{Notes}
* insert notes here *
 \subsection{ENgetlinknodes}
\subsubsection{Declaration}
\begin{lstlisting}
int ENgetlinknodes(int index, int *node1, int *node2);
\end{lstlisting}
\subsubsection{Description}
** insert description here **
\subsubsection{Arguments}
\begin{tabular}{l r p{11cm} }
int&index&*insert arg description here* \\[6pt]
int pointer&node1&*insert arg description here* \\[6pt]
int pointer&node2&*insert arg description here* \\[6pt]
\end{tabular}
\subsubsection{Returns}
Returns an error code.
\subsubsection{Notes}
* insert notes here *
 \subsection{ENgetlinkvalue}
\subsubsection{Declaration}
\begin{lstlisting}
int ENgetlinkvalue(int index, int code, float *value);
\end{lstlisting}
\subsubsection{Description}
** insert description here **
\subsubsection{Arguments}
\begin{tabular}{l r p{11cm} }
int&index&*insert arg description here* \\[6pt]
int&code&*insert arg description here* \\[6pt]
float pointer&value&*insert arg description here* \\[6pt]
\end{tabular}
\subsubsection{Returns}
Returns an error code.
\subsubsection{Notes}
* insert notes here *

\section{Retrieving information about time patterns}
 \subsection{ENgetpatternid}
\subsubsection{Declaration}
\begin{lstlisting}
int ENgetpatternid(int index, char *id);
\end{lstlisting}
\subsubsection{Description}
** insert description here **
\subsubsection{Arguments}
\begin{tabular}{l r p{11cm} }
int&index&*insert arg description here* \\[6pt]
char pointer&id&*insert arg description here* \\[6pt]
\end{tabular}
\subsubsection{Returns}
Returns an error code.
\subsubsection{Notes}
* insert notes here *
 \subsection{ENgetpatternindex}
\subsubsection{Declaration}
\begin{lstlisting}
int ENgetpatternindex(char *id, int *index);
\end{lstlisting}
\subsubsection{Description}
** insert description here **
\subsubsection{Arguments}
\begin{tabular}{l r p{11cm} }
char pointer&id&*insert arg description here* \\[6pt]
int pointer&index&*insert arg description here* \\[6pt]
\end{tabular}
\subsubsection{Returns}
Returns an error code.
\subsubsection{Notes}
* insert notes here *
 \subsection{ENgetpatternlen}
\subsubsection{Declaration}
\begin{lstlisting}
int ENgetpatternlen(int index, int *len);
\end{lstlisting}
\subsubsection{Description}
** insert description here **
\subsubsection{Arguments}
\begin{tabular}{l r p{11cm} }
int&index&*insert arg description here* \\[6pt]
int pointer&len&*insert arg description here* \\[6pt]
\end{tabular}
\subsubsection{Returns}
Returns an error code.
\subsubsection{Notes}
* insert notes here *
 \subsection{ENgetpatternvalue}
\subsubsection{Declaration}
\begin{lstlisting}
int ENgetpatternvalue(int index, int period, float *value);
\end{lstlisting}
\subsubsection{Description}
** insert description here **
\subsubsection{Arguments}
\begin{tabular}{l r p{11cm} }
int&index&*insert arg description here* \\[6pt]
int&period&*insert arg description here* \\[6pt]
float pointer&value&*insert arg description here* \\[6pt]
\end{tabular}
\subsubsection{Returns}
Returns an error code.
\subsubsection{Notes}
* insert notes here *

\section{Retrieving other network information}
 \subsection{ENgetcount}
\subsubsection{Declaration}
\begin{lstlisting}
int ENgetcount(int code, int *count);
\end{lstlisting}
\subsubsection{Description}
retrieves the number of components of a given type in the network 
\subsubsection{Arguments}
\begin{tabular}{l r p{11cm} }
int&code&component code\\[6pt]
int pointer&count&number of components in network\\[6pt]
\end{tabular}
\subsubsection{Returns}
Returns an error code.
\subsubsection{Notes}
* insert component code table here *
 \subsection{ENgetcontrol}
\subsubsection{Declaration}
\begin{lstlisting}
int ENgetcontrol(int controlIndex, int *controlType, int *linkIdx, float *setting, int *nodeIdx, float *level);
\end{lstlisting}
\subsubsection{Description}
** insert description here **
\subsubsection{Arguments}
\begin{tabular}{l r p{11cm} }
int&controlIndex&*insert arg description here* \\[6pt]
int pointer&controlType&*insert arg description here* \\[6pt]
int pointer&linkIdx&*insert arg description here* \\[6pt]
float pointer&setting&*insert arg description here* \\[6pt]
int pointer&nodeIdx&*insert arg description here* \\[6pt]
float pointer&level&*insert arg description here* \\[6pt]
\end{tabular}
\subsubsection{Returns}
Returns an error code.
\subsubsection{Notes}
* insert notes here *
 \subsection{ENgetqualtype}
\subsubsection{Declaration}
\begin{lstlisting}
int ENgetqualtype(int *qualcode, int *tracenode);
\end{lstlisting}
\subsubsection{Description}
** insert description here **
\subsubsection{Arguments}
\begin{tabular}{l r p{11cm} }
int pointer&qualcode&*insert arg description here* \\[6pt]
int pointer&tracenode&*insert arg description here* \\[6pt]
\end{tabular}
\subsubsection{Returns}
Returns an error code.
\subsubsection{Notes}
* insert notes here *
 \subsection{ENgetoption}
\subsubsection{Declaration}
\begin{lstlisting}
int ENgetoption(int code, float *value);
\end{lstlisting}
\subsubsection{Description}
** insert description here **
\subsubsection{Arguments}
\begin{tabular}{l r p{11cm} }
int&code&*insert arg description here* \\[6pt]
float pointer&value&*insert arg description here* \\[6pt]
\end{tabular}
\subsubsection{Returns}
Returns an error code.
\subsubsection{Notes}
* insert notes here *
 
\section{Setting new values for network parameters} 
 \subsection{ENsetcontrol}
\subsubsection{Declaration}
\begin{lstlisting}
int ENsetcontrol(int cindex, int ctype, int lindex, float setting, int nindex, float level);
\end{lstlisting}
\subsubsection{Description}
** insert description here **
\subsubsection{Arguments}
\begin{tabular}{l r p{11cm} }
int&cindex&*insert arg description here* \\[6pt]
int&ctype&*insert arg description here* \\[6pt]
int&lindex&*insert arg description here* \\[6pt]
float&setting&*insert arg description here* \\[6pt]
int&nindex&*insert arg description here* \\[6pt]
float&level&*insert arg description here* \\[6pt]
\end{tabular}
\subsubsection{Returns}
Returns an error code.
\subsubsection{Notes}
* insert notes here *
 \subsection{ENsetnodevalue}
\subsubsection{Declaration}
\begin{lstlisting}
int ENsetnodevalue(int index, int code, float v);
\end{lstlisting}
\subsubsection{Description}
** insert description here **
\subsubsection{Arguments}
\begin{tabular}{l r p{11cm} }
int&index&*insert arg description here* \\[6pt]
int&code&*insert arg description here* \\[6pt]
float&v&*insert arg description here* \\[6pt]
\end{tabular}
\subsubsection{Returns}
Returns an error code.
\subsubsection{Notes}
* insert notes here *
 \subsection{ENsetlinkvalue}
\subsubsection{Declaration}
\begin{lstlisting}
int ENsetlinkvalue(int index, int code, float v);
\end{lstlisting}
\subsubsection{Description}
** insert description here **
\subsubsection{Arguments}
\begin{tabular}{l r p{11cm} }
int&index&*insert arg description here* \\[6pt]
int&code&*insert arg description here* \\[6pt]
float&v&*insert arg description here* \\[6pt]
\end{tabular}
\subsubsection{Returns}
Returns an error code.
\subsubsection{Notes}
* insert notes here *
 \subsection{ENaddpattern}
\subsubsection{Declaration}
\begin{lstlisting}
int ENaddpattern(char *id);
\end{lstlisting}
\subsubsection{Description}
** insert description here **
\subsubsection{Arguments}
\begin{tabular}{l r p{11cm} }
char pointer&id&*insert arg description here* \\[6pt]
\end{tabular}
\subsubsection{Returns}
Returns an error code.
\subsubsection{Notes}
* insert notes here *
 \subsection{ENsetpattern}
\subsubsection{Declaration}
\begin{lstlisting}
int ENsetpattern(int index, float *f, int len);
\end{lstlisting}
\subsubsection{Description}
** insert description here **
\subsubsection{Arguments}
\begin{tabular}{l r p{11cm} }
int&index&*insert arg description here* \\[6pt]
float pointer&f&*insert arg description here* \\[6pt]
int&len&*insert arg description here* \\[6pt]
\end{tabular}
\subsubsection{Returns}
Returns an error code.
\subsubsection{Notes}
* insert notes here *
 \subsection{ENsetpatternvalue}
\subsubsection{Declaration}
\begin{lstlisting}
int ENsetpatternvalue(int index, int period, float value);
\end{lstlisting}
\subsubsection{Description}
** insert description here **
\subsubsection{Arguments}
\begin{tabular}{l r p{11cm} }
int&index&*insert arg description here* \\[6pt]
int&period&*insert arg description here* \\[6pt]
float&value&*insert arg description here* \\[6pt]
\end{tabular}
\subsubsection{Returns}
Returns an error code.
\subsubsection{Notes}
* insert notes here *
 \subsection{ENsetqualtype}
\subsubsection{Declaration}
\begin{lstlisting}
int ENsetqualtype(int qualcode, char *chemname, char *chemunits, char *tracenode);
\end{lstlisting}
\subsubsection{Description}
** insert description here **
\subsubsection{Arguments}
\begin{tabular}{l r p{11cm} }
int&qualcode&*insert arg description here* \\[6pt]
char pointer&chemname&*insert arg description here* \\[6pt]
char pointer&chemunits&*insert arg description here* \\[6pt]
char pointer&tracenode&*insert arg description here* \\[6pt]
\end{tabular}
\subsubsection{Returns}
Returns an error code.
\subsubsection{Notes}
* insert notes here *
 \subsection{ENsettimeparam}
\subsubsection{Declaration}
\begin{lstlisting}
int ENsettimeparam(int code, long value);
\end{lstlisting}
\subsubsection{Description}
** insert description here **
\subsubsection{Arguments}
\begin{tabular}{l r p{11cm} }
int&code&*insert arg description here* \\[6pt]
long&value&*insert arg description here* \\[6pt]
\end{tabular}
\subsubsection{Returns}
Returns an error code.
\subsubsection{Notes}
* insert notes here *
 \subsection{ENsetoption}
\subsubsection{Declaration}
\begin{lstlisting}
int ENsetoption(int code, float v);
\end{lstlisting}
\subsubsection{Description}
** insert description here **
\subsubsection{Arguments}
\begin{tabular}{l r p{11cm} }
int&code&*insert arg description here* \\[6pt]
float&v&*insert arg description here* \\[6pt]
\end{tabular}
\subsubsection{Returns}
Returns an error code.
\subsubsection{Notes}
* insert notes here *

\section{Saving and using hydraulic analysis results files} 
 \subsection{ENsavehydfile}
\subsubsection{Declaration}
\begin{lstlisting}
int ENsavehydfile(char *filename);
\end{lstlisting}
\subsubsection{Description}
** insert description here **
\subsubsection{Arguments}
\begin{tabular}{l r p{11cm} }
char pointer&filename&*insert arg description here* \\[6pt]
\end{tabular}
\subsubsection{Returns}
Returns an error code.
\subsubsection{Notes}
* insert notes here *
 \subsection{ENusehydfile}
\subsubsection{Declaration}
\begin{lstlisting}
int ENusehydfile(char *filename);
\end{lstlisting}
\subsubsection{Description}
** insert description here **
\subsubsection{Arguments}
\begin{tabular}{l r p{11cm} }
char pointer&filename&*insert arg description here* \\[6pt]
\end{tabular}
\subsubsection{Returns}
Returns an error code.
\subsubsection{Notes}
* insert notes here *
 
\section{Running a hydraulic analysis} 
 \subsection{ENsolveH}
\subsubsection{Declaration}
\begin{lstlisting}
int ENsolveH();
\end{lstlisting}
\subsubsection{Description}
** insert description here **
\subsubsection{Returns}
Returns an error code.
\subsubsection{Notes}
* insert notes here *
 \subsection{ENopenH}
\subsubsection{Declaration}
\begin{lstlisting}
int ENopenH();
\end{lstlisting}
\subsubsection{Description}
** insert description here **
\subsubsection{Returns}
Returns an error code.
\subsubsection{Notes}
* insert notes here *
 \subsection{ENinitH}
\subsubsection{Declaration}
\begin{lstlisting}
int ENinitH(int initFlag);
\end{lstlisting}
\subsubsection{Description}
** insert description here **
\subsubsection{Arguments}
\begin{tabular}{l r p{11cm} }
int&initFlag&*insert arg description here* \\[6pt]
\end{tabular}
\subsubsection{Returns}
Returns an error code.
\subsubsection{Notes}
* insert notes here *
 \subsection{ENrunH}
\subsubsection{Declaration}
\begin{lstlisting}
int ENrunH(long *currentTime);
\end{lstlisting}
\subsubsection{Description}
** insert description here **
\subsubsection{Arguments}
\begin{tabular}{l r p{11cm} }
long pointer&currentTime&*insert arg description here* \\[6pt]
\end{tabular}
\subsubsection{Returns}
Returns an error code.
\subsubsection{Notes}
* insert notes here *
 \subsection{ENnextH}
\subsubsection{Declaration}
\begin{lstlisting}
int ENnextH(long *tStep);
\end{lstlisting}
\subsubsection{Description}
** insert description here **
\subsubsection{Arguments}
\begin{tabular}{l r p{11cm} }
long pointer&tStep&*insert arg description here* \\[6pt]
\end{tabular}
\subsubsection{Returns}
Returns an error code.
\subsubsection{Notes}
* insert notes here *
 \subsection{ENcloseH}
\subsubsection{Declaration}
\begin{lstlisting}
int ENcloseH();
\end{lstlisting}
\subsubsection{Description}
** insert description here **
\subsubsection{Returns}
Returns an error code.
\subsubsection{Notes}
* insert notes here *
 
\section{Running a water quality analysis} 
 \subsection{ENsolveQ}
\subsubsection{Declaration}
\begin{lstlisting}
int ENsolveQ();
\end{lstlisting}
\subsubsection{Description}
** insert description here **
\subsubsection{Returns}
Returns an error code.
\subsubsection{Notes}
* insert notes here *
 \subsection{ENopenQ}
\subsubsection{Declaration}
\begin{lstlisting}
int ENopenQ();
\end{lstlisting}
\subsubsection{Description}
** insert description here **
\subsubsection{Returns}
Returns an error code.
\subsubsection{Notes}
* insert notes here *
 \subsection{ENinitQ}
\subsubsection{Declaration}
\begin{lstlisting}
int ENinitQ(int saveFlag);
\end{lstlisting}
\subsubsection{Description}
** insert description here **
\subsubsection{Arguments}
\begin{tabular}{l r p{11cm} }
int&saveFlag&*insert arg description here* \\[6pt]
\end{tabular}
\subsubsection{Returns}
Returns an error code.
\subsubsection{Notes}
* insert notes here *
 \subsection{ENrunQ}
\subsubsection{Declaration}
\begin{lstlisting}
int ENrunQ(long *currentTime);
\end{lstlisting}
\subsubsection{Description}
** insert description here **
\subsubsection{Arguments}
\begin{tabular}{l r p{11cm} }
long pointer&currentTime&*insert arg description here* \\[6pt]
\end{tabular}
\subsubsection{Returns}
Returns an error code.
\subsubsection{Notes}
* insert notes here *
 \subsection{ENnextQ}
\subsubsection{Declaration}
\begin{lstlisting}
int ENnextQ(long *tStep);
\end{lstlisting}
\subsubsection{Description}
** insert description here **
\subsubsection{Arguments}
\begin{tabular}{l r p{11cm} }
long pointer&tStep&*insert arg description here* \\[6pt]
\end{tabular}
\subsubsection{Returns}
Returns an error code.
\subsubsection{Notes}
* insert notes here *
 \subsection{ENstepQ}
\subsubsection{Declaration}
\begin{lstlisting}
int ENstepQ(long *timeLeft);
\end{lstlisting}
\subsubsection{Description}
** insert description here **
\subsubsection{Arguments}
\begin{tabular}{l r p{11cm} }
long pointer&timeLeft&*insert arg description here* \\[6pt]
\end{tabular}
\subsubsection{Returns}
Returns an error code.
\subsubsection{Notes}
* insert notes here *
 \subsection{ENcloseQ}
\subsubsection{Declaration}
\begin{lstlisting}
int ENcloseQ();
\end{lstlisting}
\subsubsection{Description}
** insert description here **
\subsubsection{Returns}
Returns an error code.
\subsubsection{Notes}
* insert notes here *
 
\section{Generating an output report} 
 \subsection{ENsaveH}
\subsubsection{Declaration}
\begin{lstlisting}
int ENsaveH();
\end{lstlisting}
\subsubsection{Description}
** insert description here **
\subsubsection{Returns}
Returns an error code.
\subsubsection{Notes}
* insert notes here *
 \subsection{ENsaveinpfile}
\subsubsection{Declaration}
\begin{lstlisting}
int ENsaveinpfile(char *filename);
\end{lstlisting}
\subsubsection{Description}
** insert description here **
\subsubsection{Arguments}
\begin{tabular}{l r p{11cm} }
char pointer&filename&*insert arg description here* \\[6pt]
\end{tabular}
\subsubsection{Returns}
Returns an error code.
\subsubsection{Notes}
* insert notes here *
 \subsection{ENreport}
\subsubsection{Declaration}
\begin{lstlisting}
int ENreport();
\end{lstlisting}
\subsubsection{Description}
** insert description here **
\subsubsection{Returns}
Returns an error code.
\subsubsection{Notes}
* insert notes here *
 \subsection{ENresetreport}
\subsubsection{Declaration}
\begin{lstlisting}
int ENresetreport();
\end{lstlisting}
\subsubsection{Description}
** insert description here **
\subsubsection{Returns}
Returns an error code.
\subsubsection{Notes}
* insert notes here *
 \subsection{ENsetreport}
\subsubsection{Declaration}
\begin{lstlisting}
int ENsetreport(char *reportFormat);
\end{lstlisting}
\subsubsection{Description}
** insert description here **
\subsubsection{Arguments}
\begin{tabular}{l r p{11cm} }
char pointer&reportFormat&*insert arg description here* \\[6pt]
\end{tabular}
\subsubsection{Returns}
Returns an error code.
\subsubsection{Notes}
* insert notes here *
 \subsection{ENsetstatusreport}
\subsubsection{Declaration}
\begin{lstlisting}
int ENsetstatusreport(int code);
\end{lstlisting}
\subsubsection{Description}
sets level of hydraulic status reporting
\subsubsection{Arguments}
\begin{tabular}{l r p{11cm} }
int&code&status reporting code (0, 1, or 2)\\[6pt]
\end{tabular}
\subsubsection{Returns}
Returns an error code.
\subsubsection{Notes}
* insert status reporting code table *
 \subsection{ENgeterror}
\subsubsection{Declaration}
\begin{lstlisting}
int ENgeterror(int errcode, char *errmsg, int maxLen);
\end{lstlisting}
\subsubsection{Description}
** insert description here **
\subsubsection{Arguments}
\begin{tabular}{l r p{11cm} }
int&errcode&*insert arg description here* \\[6pt]
char pointer&errmsg&*insert arg description here* \\[6pt]
int&maxLen&*insert arg description here* \\[6pt]
\end{tabular}
\subsubsection{Returns}
Returns an error code.
\subsubsection{Notes}
* insert notes here *
 \subsection{ENwriteline}
\subsubsection{Declaration}
\begin{lstlisting}
int ENwriteline(char *line);
\end{lstlisting}
\subsubsection{Description}
** insert description here **
\subsubsection{Arguments}
\begin{tabular}{l r p{11cm} }
char pointer&line&*insert arg description here* \\[6pt]
\end{tabular}
\subsubsection{Returns}
Returns an error code.
\subsubsection{Notes}
* insert notes here *

\chapter{Input File Format}
The input file for EPANET has the same format as the text file that
Windows EPANET generates from its File Export  Network command. 

It is organized in sections, where each section begins with a keyword enclosed in
brackets. The various keywords are listed in table \ref{tab:keywords}.



\begin{table}[tbp]
      \begin{tabular}{|c| l|}
	\hline
	  \multicolumn{2}{|c|}{used keywords} \\
	\hline
	\multirow{8}{*}{Network Components}
	& [TITLE]      \\
	& [JUNCTIONS]  \\
	& [RESERVOIRS] \\
	& [TANKS]      \\
	& [PIPES]      \\
	& [PUMPS]      \\
	& [VALVES]     \\
	& [EMITTERS]   \\
	\hline
	\multirow{7}{*}{System Operation}
	& [CURVES]   \\
	& [PATTERNS] \\
	& [ENERGY]   \\
	& [STATUS]   \\
	& [CONTROLS] \\
	& [RULES]    \\
	& [DEMANDS]  \\
	\hline
	\multirow{4}{*}{Water Quality}
	& [QUALITY]   \\
	& [REACTIONS] \\
	& [SOURCES]   \\
	& [MIXING]    \\
	\hline
	\multirow{3}{*}{Options and Reporting}
	& [OPTIONS]\\
	& [TIMES]  \\
	& [REPORT] \\
	\hline
	  \multicolumn{2}{|c|}{unused keywords} \\
	\hline
	
	\multirow{5}{*}{Network Map/Tags}
	& [COORDINATES]\\
	& [VERTICES]   \\
	& [LABELS]     \\
	& [BACKDROP]   \\
	& [TAGS]       \\
	\hline
  \end{tabular}

  \caption{keywords used in inpfile}
  \label{tab:keywords}

\end{table}

The order of sections is not important. However, whenever a node or link is referred
to in a section it must have already been defined in the [JUNCTIONS],
[RESERVOIRS], [TANKS], [PIPES], [PUMPS], or [VALVES] sections. Thus it is
recommended that these sections be placed first, right after the [TITLE] section. The
network map and tags sections are not used by EPANET and can be
eliminated from the file.Each section can contain one or more lines of data. Blank lines can appear anywhere
in the file and the semicolon (;) can be used to indicate that what follows on the line
is a comment, not data. A maximum of 255 characters can appear on a line. The ID
labels used to identify nodes, links, curves and patterns can be any combination of up
to 31 characters and numbers.

\section{Network Components}
\subsection{[TITLE]}
  \subsubsection{Purpose}
  Attaches a descriptive title to the network being analyzed.
  \subsubsection{Format}
  Any number of lines of text.
  \subsubsection{Remarks}
  The [TITLE] section is optional.
\subsection{[JUNCTIONS]}
  \subsubsection{Purpose}
  Defines junction nodes contained in the network.
  \subsubsection{Format}
  One line for each junction containing:
  \begin{itemize}
    \item ID label
    \item Elevation, ft (m)
    \item Base demand flow (flow units) (optional)
    \item Demand pattern ID (optional)
  \end{itemize}
  \subsubsection{Remarks}
  \begin{enumerate}
    \item A [JUNCTIONS] section with at least one junction is required.
    \item If no demand pattern is supplied then the junction demand follows the 
          Default Demand Pattern specified in the [OPTIONS] section or Pattern 1 
	  if no default pattern is specified. If the default pattern (or Pattern 1) 
	  does not exist, then the demand remains constant.
    \item Demands can also be entered in the [DEMANDS] section and include multiple
           demand categories per junction.
  \end{enumerate}
  \subsubsection{Example}
\begin{verbatim}[JUNCTIONS]
;ID Elev. Demand Pattern
;------------------------------
J1 100 50 Pat1
J2 120 10 ;Uses default demand pattern
J3 115 ;No demand at this junction
\end{verbatim}

\subsection{[RESERVOIRS]}
  \subsubsection{Purpose}
  Defines all reservoir nodes contained in the network.
  \subsubsection{Format}
    One line for each junction containing:
  \begin{itemize}
    \item ID label
    \item Head, ft (m)
    \item Head pattern ID (optional)
  \end{itemize}
  \subsubsection{Remarks}
  \begin{enumerate}
    \item Head is the hydraulic head (elevation + pressure head) of water in the reservoir.
    \item A head pattern can be used to make the reservoir head vary with time.
    \item At least one reservoir or tank must be contained in the network.
  \end{enumerate}
  \subsubsection{Example}
\begin{verbatim}[RESERVOIRS]
;ID Head Pattern
;---------------------
R1 512 ;Head stays constant
R2 120 Pat1 ;Head varies with time
\end{verbatim}

\subsection{[TANKS]}
  \subsubsection{Purpose}
  Defines all tank nodes contained in the network
  \subsubsection{Format}
    One line for each junction containing:
  \begin{itemize}
    \item ID label
    \item Bottom elevation, ft (m)
    \item Initial water level, ft (m)
    \item Minimum water level, ft (m)
    \item Maximum water level, ft (m)
    \item Nominal diameter, ft (m)
    \item Minimum volume, cubic ft (cubic meters)
    \item Volume curve ID (optional)
  \end{itemize}
  \subsubsection{Remarks}
  \begin{enumerate}
    \item Water surface elevation equals bottom elevation plus water level.
    \item Non-cylindrical tanks can be modeled by specifying a curve of volume 
          versus water depth in the [CURVES] section.
    \item If a volume curve is supplied the diameter value can be any non-zero number
    \item Minimum volume (tank volume at minimum water level) can be zero for a cylindrical
         tank or if a volume curve is supplied.
    \item A network must contain at least one tank or reservoir.
  \end{enumerate}
  \subsubsection{Example}
\begin{verbatim}[TANKS]
;ID Elev. InitLvl MinLvl MaxLvl Diam MinVol VolCurve
;-----------------------------------------------------------
;Cylindrical tank
T1 100 15 5 25 120 0
;Non-cylindrical tank with arbitrary diameter
T2 100 15 5 25 1 0 VC1
\end{verbatim}



\subsection{[PIPES]}
  \subsubsection{Purpose}
  Defines all pipe links contained in the network.
  \subsubsection{Format}
  One line for each junction containing:
  \begin{itemize}
    \item ID label of pipe
    \item ID of start node
    \item ID of end node
    \item Length, ft (m)
    \item Diameter, inches (mm)
    \item Roughness coefficient
    \item Minor loss coefficient
    \item Status (OPEN, CLOSED, or CV)
  \end{itemize}
  \subsubsection{Remarks}
  \begin{enumerate}
    \item Roughness coefficient is unitless for the Hazen-Williams and Chezy-Manning 
          head loss formulas and has units of millifeet (mm) for the Darcy-Weisbach 
	  formula. Choice of head loss formula is supplied in the [OPTIONS] section.
    \item Setting status to CV means that the pipe contains a check valve restricting
          flow to one direction.
    \item If minor loss coefficient is 0 and pipe is OPEN then these two items can
          be dropped form the input line.
  \end{enumerate}
  \subsubsection{Example}
\begin{verbatim}[PIPES]
;ID Node1 Node2 Length Diam. Roughness Mloss Status
;-------------------------------------------------------------
P1  J1    J2    1200   12    120       0.2   OPEN
P2  J3    J2     600    6    110       0     CV
P3  J1    J10   1000   12    120
\end{verbatim}

\subsection{[PUMPS]}
  \subsubsection{Purpose}
  Defines all pump links contained in the network.
  \subsubsection{Format}
  One line for each junction containing:
  \begin{itemize}
    \item ID label of pump
    \item ID of start node
    \item ID of end node
    \item Keyword and Value (can be repeated)
  \end{itemize}
  \subsubsection{Remarks}
  \begin{enumerate}
    \item Keywords consists of:
    \begin{itemize}
        \item[-] POWER  power value for constant energy pump, hp (kW)
        \item[-] HEAD  ID of curve that describes head versus flow for the pump
        \item[-] SPEED  relative speed setting (normal speed is 1.0, 0 means pump is off)
        \item[-] PATTERN  ID of time pattern that describes how speed setting varies with time
    \end{itemize}
    \item Either POWER or HEAD must be supplied for each pump. The other keywords are optional.
  \end{enumerate}
  \subsubsection{Example}
\begin{verbatim}[PUMPS]
;ID   Node1 Node2 Properties
;---------------------------------------------
Pump1 N12   N32   HEAD Curve1
Pump2 N121  N55   HEAD Curve1   SPEED 1.2
Pump3 N22   N23   POWER 100
\end{verbatim}


\begin{table}[tbp]
  \begin{tabular}{|c|l|l|}
\hline
Valve& Type& Setting\\
\hline
PRV& pressure reducing valve  & Pressure, psi (m)\\
PSV& pressure sustaining valve& Pressure, psi (m)\\
PBV& pressure breaker valve   & Pressure, psi (m)\\
FCV& flow control valve       & Flow (flow units)\\
TCV& throttle control valve   & Loss Coefficient \\
GPV& general purpose valve    & ID of head loss curve\\
\hline
\end{tabular}

  \caption{Valve types and settings}
  \label{tab:contolvalves}
\end{table}



\subsection{[VALVES]}
  \subsubsection{Purpose}
  Defines all control valve links contained in the network.
  \subsubsection{Format}
  One line for each junction containing:
  \begin{itemize}
    \item ID label of valve
    \item ID of start node
    \item ID of end node
    \item Diameter, inches (mm)
    \item Valve type
    \item Valve setting
    \item Minor loss coefficient
  \end{itemize}
  \subsubsection{Remarks}
  \begin{enumerate}
    \item Valve types and settings see Table \ref{tab:contolvalves}
    \item Shutoff valves and check valves are considered to be part of a pipe,
          not a separate control valve component (see [PIPES])
  \end{enumerate}


\subsection{[EMITTERS]}
  \subsubsection{Purpose}
  \subsubsection{Format}
  One line for each junction containing:
  \begin{itemize}
    \item
  \end{itemize}


  

  
  



\section{System Operation}
  ... todo
\section{Water Quality}
  ... todo
\section{Options and Reporting}
  ... todo
\section{Network Map/Tags}
  ... todo
  
\chapter{Binary Output File Format}
If a third file name is supplied to the command line that runs EPANET then the
results for all parameters for all nodes and links for all reporting time periods will be
saved to this file in a special binary format. This file can be used for special postprocessing
purposes. Data written to the file are 4-byte integers, 4-byte floats, or
fixed-size strings whose size is a multiple of 4 bytes. This allows the file to be
divided conveniently into 4-byte records. The file consists of four sections of the
following sizes in bytes:
\begin{tabular}{|l|l|}
\hline
Section &Size in bytes\\
\hline
Prolog& $852 + 20\cdot Nnodes + 36\cdot Nlinks + 8\cdot Ntanks$   \\
Energy Use& $28\cdot Npumps + 4 $   \\
Extended Period& $(16\cdot Nnodes + 32\cdot Nlinks)*Nperiods$ \\
Epilog& 28 \\
\hline
\end{tabular}

\begin{tabular}{l l }
$Nnodes$ & number of nodes (junctions + reservoirs + tanks) \\
$Nlinks$ & number of links (pipes + pumps + valves)\\
$Ntanks$ & number of tanks and reservoirs\\
$Npumps$ & number of pumps\\
$Nperiods$& number of reporting periods\\
\end{tabular}

All of these counts are themselves written to the file's Prolog or Epilog sections.

\section{Prolog}

\begin{tabular}{|l|c|r|}
\hline
Item& Type & Number of Bytes  \\
\hline
Magic Number ( = 516114521)& Integer& 4 \\
Version& Integer &4 \\
Number of Nodes (Junctions + Reservoirs + Tanks) & Integer &4\\
Number of Reservoirs \& Tanks &Integer &4\\
Number of Links (Pipes + Pumps + Valves)&Integer& 4\\
Number of Pumps& Integer& 4\\
Number of Valves& Integer& 4\\
Water Quality Option&Integer& 4\\
Index of Node for Source Tracing& Integer& 4\\
Flow Units Option&Integer &4\\
Pressure Units Option & Integer & 4 \\
Statistics Flag& Integer & 4\\
Reporting Start Time (seconds) & Integer & 4 \\
Reporting Time Step (seconds) & Integer & 4 \\
Simulation Duration (seconds) & Integer & 4 \\
Problem Title (1st line) & Char & 80 \\
Problem Title (2nd line) & Char & 80 \\
Problem Title (3rd line) & Char & 80 \\
Name of Input File & Char & 260   \\
Name of Report File & Char & 260  \\
Name of Chemical & Char & 32      \\
Chemical Concentration Units & Char & 32 \\
ID Label of Each Node & Char & $32 \cdot Nnodes$ \\
ID Label of Each Link & Char & $32 \cdot Nlinks$ \\
Index of Start Node of Each Link & Integer & $4 \cdot Nlinks$ \\
Index of End Node of Each Link & Integer & $4 \cdot Nlinks$   \\
Type Code of Each Link & Integer & $4 \cdot Nlinks$           \\
Node Index of Each Tank & Integer & $4 \cdot Ntanks$          \\
Cross-Sectional Area of Each Tank & Float & $4 \cdot Ntanks$  \\
Elevation of Each Node & Float & $4 \cdot Nnodes$             \\
Length of Each Link & Float & $4 \cdot Nlinks$                \\
Diameter of Each Link & Float & $4 \cdot Nlinks$              \\
\hline
\end{tabular}


\appendix
\chapter{Example input file}
\begin{verbatim}
[TITLE]
EPANET TUTORIAL
[JUNCTIONS]
;ID Elev Demand
;------------------
2     0    0
3   710  650
4   700  150
5   695  200
6   700  150
[RESERVOIRS]
;ID Head
;---------
1   700
[TANKS]
;ID Elev InitLvl MinLvl MaxLvl Diam Volume
;-----------------------------------------------
7   850  5       0      15     70   0
[PIPES]
;ID Node1 Node2 Length Diam Roughness
;-----------------------------------------
1   2     3     3000   12   100
2   3     6     5000   12   100
3   3     4     5000    8   100
4   4     5     5000    8   100
5   5     6     5000    8   100
6   6     7     7000   10   100
[PUMPS]
;ID Node1 Node2 Parameters
;---------------------------------
7   1     2     HEAD 1
[PATTERNS]
;ID Multipliers
;-----------------------
1 0.5 1.3 1 1.2
[CURVES]
;ID X-Value Y-Value
;--------------------
1   1000    200
[QUALITY]
;Node InitQual
;-------------
1 1
[REACTIONS]
Global Bulk -1
Global Wall 0
[TIMES]
Duration 24:00
Hydraulic Timestep 1:00
Quality Timestep 0:05
Pattern Timestep 6:00
[REPORT]
Page 55
Energy Yes
Nodes All
Links All
[OPTIONS]
Units GPM
Headloss H-W
Pattern 1
Quality Chlorine mg/L
Tolerance 0.01
[END]
\end{verbatim}

\chapter{Error Messages}
\begin{longtable}{ c p{12cm}}
\hline
101& An analysis was terminated due to insufficient memory available.\\[6pt]

110& An analysis was terminated because the network hydraulic equations could
not be solved. Check for portions of the network not having any physical
links back to a tank or reservoir or for unreasonable values for network input
data.\\[6pt]

200& One or more errors were detected in the input data. The nature of the error
will be described by the 200-series error messages listed below.\\[6pt]

201& There is a syntax error in a line of the input file created from your network
data. This is most likely to have occurred in .INP text created by a user
outside of EPANET.\\[6pt]

202& An illegal numeric value was assigned to a property.\\[6pt]

203& An object refers to undefined node. \\[6pt]

204& An object refers to an undefined link.\\[6pt]

205& An object refers to an undefined time pattern.\\[6pt]

206& An object refers to an undefined curve.\\[6pt]

207& An attempt is made to control a check valve. Once a pipe is assigned a Check
Valve status with the Property Editor, its status cannot be changed by either
simple or rule-based controls. \\[6pt]

208& Reference was made to an undefined node. This could occur in a control
statement for example. \\[6pt]

209& An illegal value was assigned to a node property. \\[6pt]

210& Reference was made to an undefined link. This could occur in a control
statement for example.  \\[6pt]

211& An illegal value was assigned to a link property.   \\[6pt]

212& A source tracing analysis refers to an undefined trace node.\\[6pt]

213& An analysis option has an illegal value (an example would be a negative time
step value).\\[6pt]

214& There are too many characters in a line read from an input file. The lines in
the .INP file are limited to 255 characters.\\[6pt]

215& Two or more nodes or links share the same ID label. \\[6pt]

216& Energy data were supplied for an undefined pump. \\[6pt]

217& Invalid energy data were supplied for a pump. \\[6pt]

219& A valve is illegally connected to a reservoir or tank. A PRV, PSV or FCV
cannot be directly connected to a reservoir or tank. Use a length of pipe to
separate the two.      \\[6pt]

220& A valve is illegally connected to another valve. PRVs cannot share the same
downstream node or be linked in series, PSVs cannot share the same
upstream node or be linked in series, and a PSV cannot be directly connected
to the downstream node of a PRV.\\[6pt]

221& A rule-based control contains a misplaced clause.\\[6pt]

223& There are not enough nodes in the network to analyze. A valid network must
contain at least one tank/reservoir and one junction node.\\[6pt]

224& There is not at least one tank or reservoir in the network.\\[6pt]

225& Invalid lower/upper levels were specified for a tank (e.g., the lower lever is
higher than the upper level).\\[6pt]

226& No pump curve or power rating was supplied for a pump. A pump must either
be assigned a curve ID in its Pump Curve property or a power rating in its
Power property. If both properties are assigned then the Pump Curve is used.\\[6pt]

227& A pump has an invalid pump curve. A valid pump curve must have
decreasing head with increasing flow. \\[6pt]

230& A curve has non-increasing X-values. \\[6pt]

233& A node is not connected to any links.\\[6pt]

302& The system cannot open the temporary input file. Make sure that the
EPANET Temporary Folder selected has write privileges assigned to it (see
Section 4.9). \\[6pt]

303& The system cannot open the status report file. See Error 302.\\[6pt]

304& The system cannot open the binary output file. See Error 302.  \\[6pt]

308& Could not save results to file. This can occur if the disk becomes full.\\[6pt]

309& Could not write results to report file. This can occur if the disk becomes full.\\[6pt]
\hline
\end{longtable}


\end{document}
