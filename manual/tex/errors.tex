\chapter{Error Messages}
\begin{longtable}{ c p{12cm}}
\hline
101& An analysis was terminated due to insufficient memory available.\\[6pt]

110& An analysis was terminated because the network hydraulic equations could
not be solved. Check for portions of the network not having any physical
links back to a tank or reservoir or for unreasonable values for network input
data.\\[6pt]

200& One or more errors were detected in the input data. The nature of the error
will be described by the 200-series error messages listed below.\\[6pt]

201& There is a syntax error in a line of the input file created from your network
data. This is most likely to have occurred in .INP text created by a user
outside of EPANET.\\[6pt]

202& An illegal numeric value was assigned to a property.\\[6pt]

203& An object refers to undefined node. \\[6pt]

204& An object refers to an undefined link.\\[6pt]

205& An object refers to an undefined time pattern.\\[6pt]

206& An object refers to an undefined curve.\\[6pt]

207& An attempt is made to control a check valve. Once a pipe is assigned a Check
Valve status with the Property Editor, its status cannot be changed by either
simple or rule-based controls. \\[6pt]

208& Reference was made to an undefined node. This could occur in a control
statement for example. \\[6pt]

209& An illegal value was assigned to a node property. \\[6pt]

210& Reference was made to an undefined link. This could occur in a control
statement for example.  \\[6pt]

211& An illegal value was assigned to a link property.   \\[6pt]

212& A source tracing analysis refers to an undefined trace node.\\[6pt]

213& An analysis option has an illegal value (an example would be a negative time
step value).\\[6pt]

214& There are too many characters in a line read from an input file. The lines in
the .INP file are limited to 255 characters.\\[6pt]

215& Two or more nodes or links share the same ID label. \\[6pt]

216& Energy data were supplied for an undefined pump. \\[6pt]

217& Invalid energy data were supplied for a pump. \\[6pt]

219& A valve is illegally connected to a reservoir or tank. A PRV, PSV or FCV
cannot be directly connected to a reservoir or tank. Use a length of pipe to
separate the two.      \\[6pt]

220& A valve is illegally connected to another valve. PRVs cannot share the same
downstream node or be linked in series, PSVs cannot share the same
upstream node or be linked in series, and a PSV cannot be directly connected
to the downstream node of a PRV.\\[6pt]

221& A rule-based control contains a misplaced clause.\\[6pt]

223& There are not enough nodes in the network to analyze. A valid network must
contain at least one tank/reservoir and one junction node.\\[6pt]

224& There is not at least one tank or reservoir in the network.\\[6pt]

225& Invalid lower/upper levels were specified for a tank (e.g., the lower lever is
higher than the upper level).\\[6pt]

226& No pump curve or power rating was supplied for a pump. A pump must either
be assigned a curve ID in its Pump Curve property or a power rating in its
Power property. If both properties are assigned then the Pump Curve is used.\\[6pt]

227& A pump has an invalid pump curve. A valid pump curve must have
decreasing head with increasing flow. \\[6pt]

230& A curve has non-increasing X-values. \\[6pt]

233& A node is not connected to any links.\\[6pt]

302& The system cannot open the temporary input file. Make sure that the
EPANET Temporary Folder selected has write privileges assigned to it (see
Section 4.9). \\[6pt]

303& The system cannot open the status report file. See Error 302.\\[6pt]

304& The system cannot open the binary output file. See Error 302.  \\[6pt]

308& Could not save results to file. This can occur if the disk becomes full.\\[6pt]

309& Could not write results to report file. This can occur if the disk becomes full.\\[6pt]
\hline
\end{longtable}
