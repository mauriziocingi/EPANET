\subsection{[TITLE]}
  \subsubsection{Purpose}
  Attaches a descriptive title to the network being analyzed.
  \subsubsection{Format}
  Any number of lines of text.
  \subsubsection{Remarks}
  The [TITLE] section is optional.
\subsection{[JUNCTIONS]}
  \subsubsection{Purpose}
  Defines junction nodes contained in the network.
  \subsubsection{Format}
  One line for each junction containing:
  \begin{itemize}
    \item ID label
    \item Elevation, ft (m)
    \item Base demand flow (flow units) (optional)
    \item Demand pattern ID (optional)
  \end{itemize}
  \subsubsection{Remarks}
  \begin{enumerate}
    \item A [JUNCTIONS] section with at least one junction is required.
    \item If no demand pattern is supplied then the junction demand follows the 
          Default Demand Pattern specified in the [OPTIONS] section or Pattern 1 
	  if no default pattern is specified. If the default pattern (or Pattern 1) 
	  does not exist, then the demand remains constant.
    \item Demands can also be entered in the [DEMANDS] section and include multiple
           demand categories per junction.
  \end{enumerate}
  \subsubsection{Example}
\begin{verbatim}[JUNCTIONS]
;ID Elev. Demand Pattern
;------------------------------
J1 100 50 Pat1
J2 120 10 ;Uses default demand pattern
J3 115 ;No demand at this junction
\end{verbatim}

\subsection{[RESERVOIRS]}
  \subsubsection{Purpose}
  Defines all reservoir nodes contained in the network.
  \subsubsection{Format}
    One line for each junction containing:
  \begin{itemize}
    \item ID label
    \item Head, ft (m)
    \item Head pattern ID (optional)
  \end{itemize}
  \subsubsection{Remarks}
  \begin{enumerate}
    \item Head is the hydraulic head (elevation + pressure head) of water in the reservoir.
    \item A head pattern can be used to make the reservoir head vary with time.
    \item At least one reservoir or tank must be contained in the network.
  \end{enumerate}
  \subsubsection{Example}
\begin{verbatim}[RESERVOIRS]
;ID Head Pattern
;---------------------
R1 512 ;Head stays constant
R2 120 Pat1 ;Head varies with time
\end{verbatim}

\subsection{[TANKS]}
  \subsubsection{Purpose}
  Defines all tank nodes contained in the network
  \subsubsection{Format}
    One line for each junction containing:
  \begin{itemize}
    \item ID label
    \item Bottom elevation, ft (m)
    \item Initial water level, ft (m)
    \item Minimum water level, ft (m)
    \item Maximum water level, ft (m)
    \item Nominal diameter, ft (m)
    \item Minimum volume, cubic ft (cubic meters)
    \item Volume curve ID (optional)
  \end{itemize}
  \subsubsection{Remarks}
  \begin{enumerate}
    \item Water surface elevation equals bottom elevation plus water level.
    \item Non-cylindrical tanks can be modeled by specifying a curve of volume 
          versus water depth in the [CURVES] section.
    \item If a volume curve is supplied the diameter value can be any non-zero number
    \item Minimum volume (tank volume at minimum water level) can be zero for a cylindrical
         tank or if a volume curve is supplied.
    \item A network must contain at least one tank or reservoir.
  \end{enumerate}
  \subsubsection{Example}
\begin{verbatim}[TANKS]
;ID Elev. InitLvl MinLvl MaxLvl Diam MinVol VolCurve
;-----------------------------------------------------------
;Cylindrical tank
T1 100 15 5 25 120 0
;Non-cylindrical tank with arbitrary diameter
T2 100 15 5 25 1 0 VC1
\end{verbatim}



\subsection{[PIPES]}
  \subsubsection{Purpose}
  Defines all pipe links contained in the network.
  \subsubsection{Format}
  One line for each junction containing:
  \begin{itemize}
    \item ID label of pipe
    \item ID of start node
    \item ID of end node
    \item Length, ft (m)
    \item Diameter, inches (mm)
    \item Roughness coefficient
    \item Minor loss coefficient
    \item Status (OPEN, CLOSED, or CV)
  \end{itemize}
  \subsubsection{Remarks}
  \begin{enumerate}
    \item Roughness coefficient is unitless for the Hazen-Williams and Chezy-Manning 
          head loss formulas and has units of millifeet (mm) for the Darcy-Weisbach 
	  formula. Choice of head loss formula is supplied in the [OPTIONS] section.
    \item Setting status to CV means that the pipe contains a check valve restricting
          flow to one direction.
    \item If minor loss coefficient is 0 and pipe is OPEN then these two items can
          be dropped form the input line.
  \end{enumerate}
  \subsubsection{Example}
\begin{verbatim}[PIPES]
;ID Node1 Node2 Length Diam. Roughness Mloss Status
;-------------------------------------------------------------
P1  J1    J2    1200   12    120       0.2   OPEN
P2  J3    J2     600    6    110       0     CV
P3  J1    J10   1000   12    120
\end{verbatim}

\subsection{[PUMPS]}
  \subsubsection{Purpose}
  Defines all pump links contained in the network.
  \subsubsection{Format}
  One line for each junction containing:
  \begin{itemize}
    \item ID label of pump
    \item ID of start node
    \item ID of end node
    \item Keyword and Value (can be repeated)
  \end{itemize}
  \subsubsection{Remarks}
  \begin{enumerate}
    \item Keywords consists of:
    \begin{itemize}
        \item[-] POWER  power value for constant energy pump, hp (kW)
        \item[-] HEAD  ID of curve that describes head versus flow for the pump
        \item[-] SPEED  relative speed setting (normal speed is 1.0, 0 means pump is off)
        \item[-] PATTERN  ID of time pattern that describes how speed setting varies with time
    \end{itemize}
    \item Either POWER or HEAD must be supplied for each pump. The other keywords are optional.
  \end{enumerate}
  \subsubsection{Example}
\begin{verbatim}[PUMPS]
;ID   Node1 Node2 Properties
;---------------------------------------------
Pump1 N12   N32   HEAD Curve1
Pump2 N121  N55   HEAD Curve1   SPEED 1.2
Pump3 N22   N23   POWER 100
\end{verbatim}


\begin{table}[tbp]
  \begin{tabular}{|c|l|l|}
\hline
Valve& Type& Setting\\
\hline
PRV& pressure reducing valve  & Pressure, psi (m)\\
PSV& pressure sustaining valve& Pressure, psi (m)\\
PBV& pressure breaker valve   & Pressure, psi (m)\\
FCV& flow control valve       & Flow (flow units)\\
TCV& throttle control valve   & Loss Coefficient \\
GPV& general purpose valve    & ID of head loss curve\\
\hline
\end{tabular}

  \caption{Valve types and settings}
  \label{tab:contolvalves}
\end{table}



\subsection{[VALVES]}
  \subsubsection{Purpose}
  Defines all control valve links contained in the network.
  \subsubsection{Format}
  One line for each junction containing:
  \begin{itemize}
    \item ID label of valve
    \item ID of start node
    \item ID of end node
    \item Diameter, inches (mm)
    \item Valve type
    \item Valve setting
    \item Minor loss coefficient
  \end{itemize}
  \subsubsection{Remarks}
  \begin{enumerate}
    \item Valve types and settings see Table \ref{tab:contolvalves}
    \item Shutoff valves and check valves are considered to be part of a pipe,
          not a separate control valve component (see [PIPES])
  \end{enumerate}


\subsection{[EMITTERS]}
  \subsubsection{Purpose}
  \subsubsection{Format}
  One line for each junction containing:
  \begin{itemize}
    \item
  \end{itemize}


  
