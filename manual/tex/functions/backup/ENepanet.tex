\subsection{ENepanet}
\subsubsection{Declaration}
\begin{lstlisting}
int ENepanet( char* f1, char* f2, char* f3, void (*) (vfunc) )
\end{lstlisting}

\subsubsection{Description}
Runs a complete EPANET simulation.
\subsubsection{Arguments}
\begin{tabular}{ r p{11cm} }
f1&name of the input file \\[6pt]
f2&name of an output report file \\[6pt]
f3&name of an optional binary output file\\[6pt]
vfunc& pointer to a user-supplied function which accepts a character string as its argument.\\[6pt]
\end{tabular}

 
 
  
\subsubsection{Returns}

 Returns an error code.  
 
  

\subsubsection{Notes}

 ENepanet is a stand-alone function and does not interact with any of the other functions in the toolkit.
 If there is no need to save EPANET's binary output file then f3 can be an empty string ("").
 The vfunc function pointer allows the calling program to display a progress message generated by EPANET 
 during its computations. A typical function for a console application might look as follows:
\begin{lstlisting}
void  writecon(char *s)
 {
    puts(s);
 }
\end{lstlisting}

  
 and somewhere in the calling program the following declarations would appear:  
 
\begin{lstlisting}
 void (* vfunc) (char *);
 vfunc = writecon;
 ENepanet(f1,f2,f3,vfunc);
 \end{lstlisting}

  
 If such a function is not desired then this argument should be NULL (NIL for Delphi/Pascal, 
 VBNULLSTRING for Visual Basic).
 ENepanet is used mainly to link the EPANET engine to third-party user interfaces that build network 
 input files and display the results of a network analysis.
 
